\documentclass[11pt,letterpaper,boxed]{hmcpset}

\usepackage{savetrees}
\usepackage[pdftex]{graphicx}
\usepackage{amsmath}
\usepackage{amssymb}
\usepackage{multicol}

\name{Name: \underline{\hspace{5cm}}}
\class{CS159: Natural Language Processing}
\assignment{PReview \#2}
\duedate{Due in class Feb 2, 2015}

\begin{document}

\section*{Review} 
\begin{problem}
Write a function that takes a text and generates all the bigrams in the text. It should then pick a random word in the text and iteratively
pick a following word by randomly selecting a following word from the bigrams.

Generate 20-word "samples" for your choice of 2 texts from the book texts. On your worksheet, specify which texts you used, any additional cleanup you did, and the resulting automatically-generated texts. 
\end{problem}

\begin{solution}
\vspace{6cm}
\end{solution}
 
\begin{problem}
Tag sets commonly have separate parts of speech for the comparative ("-er") and superlative ("-est") forms of adjectives. For example, the Penn Treebank Set identifies the following parts of speech: 
\begin{description}
    \item[JJ] Adjective or ordinal number (third, regrettable, happy, exciting)
    \item[JJR] Comparative adjective (fitter, happier)
    \item[JJS] Superlative adjective (happiest, creepiest, proudest)
\end{description}

Do these categories match our definition of parts of speech as being categories of words that can show up in the same place? Give a context where a JJ would work, but a JJR or a JJS would not. Similarly, give a context where only a JJR would work, and one where only a JJS would work.

(You can use the nltk similar() and common\_contexts() functions for ideas if you're stuck).
\end{problem}
\begin{solution}
\vspace{6cm}
\end{solution}

\pagebreak

\section*{Preview} 

\begin{problem}
NLTK 2.4: Read in the texts of the State of the Union addresses, using the state\_union corpus reader. Count occurrences of men, women, and people in each document. What has happened to the usage of these words over time?
\end{problem}
\begin{solution}
\vspace{10cm}
\end{solution}

\begin{problem}
NLTK 3.7: Write regular expressions to match the following classes of
strings:
\begin{enumerate}
   \item  A single determiner (assume that a, an, and the are the only determiners).
    \item An arithmetic expression using integers, addition, and multiplication, such as 2*3+8.
\end{enumerate}
\end{problem}

\begin{solution}
\begin{enumerate}
    \item \vspace{2cm}
    \item \vspace{2cm}
\end{enumerate}
\end{solution}

\end{document}