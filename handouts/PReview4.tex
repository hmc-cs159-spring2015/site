\documentclass[11pt,letterpaper,boxed]{hmcpset}

\usepackage{savetrees}
\usepackage[pdftex]{graphicx}
\usepackage{amsmath}
\usepackage{amssymb}
\usepackage{multicol}
\usepackage[utf8]{inputenc}

\DeclareUnicodeCharacter{2014}{\dash}

\name{Name: \underline{\hspace{5cm}}}
\class{CS159: Natural Language Processing}
\assignment{PReview \#4}
\duedate{Due in class Feb 16, 2015}

\begin{document}

\section*{Review} 
\begin{problem}
If you randomly pick a word from the Brown corpus ``mystery''
category, what is the probability that the word is at least six
letters long? What if you pick a word from the ``hobbies'' category? The ``government'' category?

Use the fact that there are 57169 words in the mystery category, 82345
words in the hobbies category, and 70117 words in the government
category to calculate the marginal probability of a word chosen at
random being at least six letters long if we combined all three of the
above categories. 
\end{problem}

\begin{solution}
\vspace{6cm}
\end{solution}
 
\begin{problem}
Again  using the Brown corpus ``government'' category, what is the
probability that a word $w_i$ is at least six letters long given that
the previous word $w_{i-1}$ is at least six letters long? Given that
result, what can you conclude about the (in)dependence of the lengths
of consecutive words in that data set?
\end{problem}
\begin{solution}
\vspace{6cm}
\end{solution}

\pagebreak

\section*{Preview} 

\begin{problem}
Suppose that we have the following data points that we want to cluster
with the k-means algorithm:
\begin{tabular}{|l|l|l|l|}
\hline
\textbf{x} & \textbf{y} & assignment 1 & assignment 2 \\
\hline
1 & 4 & &  \\
\hline  
 2 & 5 & & \\
\hline  
 2 & 0 & & \\
\hline  
 1 & 8 & &  \\
\hline  
 0 & 2 & & \\
\hline  
 3 & 2 & & \\
\hline  
 3 & 5 & & \\
\hline  
 4 & 6 & & \\
\hline  
 5 & 4 & & \\
\hline
\end{tabular}

For clustering into two groups, we will have two means. Suppose that
we initialize them at $(1,0)$ and $(5,6)$:

\begin{tabular}{|l|l|l|}
\hline 
\textbf{iteration} & center 1 & center 2 \\
\hline
0 & (1,0) & (5,6) \\
\hline  
 1 & & \\
\hline  
 2 & & \\
\hline  
 3 & & \\
\hline
\end{tabular}

Step through three iterations of the k-means algorithm. In the top
table, indicate whether each point is assigned to cluster 1 or cluster
2 in each iteration. In the bottom table, indicate what the
newly-computed means are for each cluster. Use the rest of this page
to show enough of your work for us to understand how you got your results.
\end{problem}
\begin{solution}
\vspace{10cm}
\end{solution}

\end{document}