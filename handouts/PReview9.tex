\documentclass[11pt,letterpaper,boxed]{hmcpset}

\usepackage{savetrees}
\usepackage[pdftex]{graphicx}
\usepackage{amsmath}
\usepackage{amssymb}
\usepackage{multicol}
\usepackage{hyperref}

\name{Name: \underline{\hspace{5cm}}}
\class{CS159: Natural Language Processing}
\assignment{PReview \#9}
\duedate{Due in class Apr 15, 2015}

\begin{document}

\section*{Review} 
 
\begin{problem}
M\&S Exercise 10.17: Most of the triggering environments in Brill
(1995a) refer to preceding context. Why? Would you expect the same
tendency for languages other than English?
\end{problem}
\begin{solution}
\vspace{8cm}
\end{solution}

\begin{problem}
M\&S Exercise 10.18: The set of possible triggering environments for
words and tags is different in (Brill 1995a). For example, ``one of
the three preceding tags is X'' is admissible as a triggering
environment, but not ``one of the three preceding words is X.'' What
might be the reason for this difference? Consider the differences
between the sizes of the search spaces for words and tags. 
\end{problem}
\begin{solution}
\vspace{8cm}
\end{solution}

\pagebreak

\section*{Preview} 
The sentence ``Time flies like an arrow'' can be generated with the
following Context Free Grammar:
\begin{multicols}{2}

\begin{align*}
    \text{S} & \to \text{NP VP} \\
    \text{NP} & \to \text{Noun} | \text{Det Noun} \\
    \text{VP} & \to \text{Verb PP} \\
    \text{PP} & \to \text{Prep NP} \\
\end{align*}

\begin{align*}
    \text{Prep} & \to like \\
    \text{Det} & \to an \\
    \text{Verb} & \to flies \\
    \text{Noun} & \to Time | arrow \\
\end{align*}    

\end{multicols}

\begin{problem}
Draw the tree for the resulting structure. You can draw the tree by
hand, or use an online tool like \url{http://mshang.ca/syntree/} to
generate an image for you.
\end{problem}
\begin{solution}
\vspace{8cm}
\end{solution}

\begin{problem}
List any new or modified rules we would need in our grammar to allow for two possible
parses of the sentence ``Fruit flies like a banana''.
\end{problem}
\begin{solution}
\vspace{6cm}
\end{solution}

\end{document}
